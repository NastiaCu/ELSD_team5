\chapter*{Conclusions}
In conclusion, this project aims to create a DSL for sound processing that provides a more intuitive way to express sound processing algorithms. DSLs are designed with a specific focus in mind and can provide easy-to-use tools. By creating a DSL, programmers have the opportunity to customize the tool to their specific needs, which can significantly speed up the development process for larger projects. Compared to General Purpose Language, a Domain Specific Language can provide a higher level of abstraction, which can make them more approachable to non-programmers or those with less programming experience.

This language was developed using the ANTLR (Another Tool for Language Recognition) framework, a widely utilized tool for designing languages from the ground up. This allowed for the creation of a syntax that is both expressive and easy to understand.

The primary objective of the project was to engineer a DSL that catered specifically to the needs of sound processing. This was accomplished by conducting an in-depth analysis of the sound processing domain, identifying common tasks, and integrating these insights into the language design. The resulting DSL allows programmers to express sound processing algorithms more intuitively, which can significantly expedite the development process, especially for larger projects.

Another aim of the project was to broaden access to sound processing. The high level of abstraction offered by the DSL makes it approachable, even to individuals with limited programming experience. It was hoped that by simplifying the programming process, the entry barrier to the sound processing field would be lowered.

In comparison to general-purpose languages, which are often complex and require a steep learning curve, the DSL developed for this project is tailored specifically to sound processing. This focus allows users to express complex tasks with less code and in a more readable way, thanks to the use of ANTLR in creating lexer and parser rules for a unique, expressive, and easily parsed grammar.

Looking to the future, it's anticipated that the DSL will continue to evolve in response to community feedback. The project's success isn't solely dependent on the initial implementation but also on the ongoing commitment to refine the language based on user needs. As sound processing techniques and technologies continue to advance, the DSL must also develop to stay a relevant and efficient tool in the field.