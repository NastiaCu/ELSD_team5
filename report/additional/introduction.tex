\chapter*{Introduction}
The project involves the development of a specialized language for sound processing, with the aim of providing practitioners and researchers in the field with a more intuitive and concise way of expressing algorithms for sound processing. A thorough Domain Analysis will be conducted to understand the unique needs and requirements of the sound processing domain, which will involve researching various techniques and approaches used in the field, as well as analyzing the characteristics and constraints of different types of sound data. 

In addition to the research, it will be utilized existing resources and documentation in the sound processing domain to inform the Domain Analysis. This may include academic papers, industry reports, and online communities and forums dedicated to sound processing.

The team's motivation for this project stems from the passion for sound processing, and the desire to make a meaningful contribution to the field by developing a language that can help to streamline and improve sound processing techniques.The language has the potential to make sound processing more accessible and intuitive for practitioners and researchers, and to contribute to the advancement of the field.

The goal of the Domain Analysis is to gain a deep understanding of the unique challenges and opportunities of the sound processing domain, and to identify the key requirements and constraints that the language must meet in order to be successful. By utilizing a range of resources and tools, and drawing on our own expertise and experience in the field, we are confident that we can develop a language that meets the needs of the sound processing community, and has the potential to make a significant contribution to the field.

To ensure that the language is effective and widely adopted by the sound processing community, it will be taken into consideration the target audience and their specific needs. It will be conducted a thorough analysis of the target audience, which will include professionals and researchers in the sound processing domain, as well as students and enthusiasts who are interested in learning more about sound processing.

The goal of this analysis is to gain a deeper understanding of the challenges and pain points that the target audience faces in their work, and to identify the specific features and functionality that they require in a sound processing language. By taking a user-centered approach, we can ensure that the language is designed with the needs of the target audience in mind, and is therefore more likely to be widely adopted and used.

In addition to considering the needs of the target audience, it will also be analyzed the existing competition and alternatives in the sound processing domain. By understanding the strengths and weaknesses of existing solutions, we can ensure that the language provides a unique value proposition and addresses the key pain points of practitioners and researchers in the field.

Overall, the development of the specialized sound processing language is aimed at providing practitioners and researchers in the field with a more intuitive and concise way of expressing algorithms for sound processing. By designing a language that is tailored to the unique needs and requirements of the sound processing domain, we aim to streamline and improve sound processing techniques, making them more accessible and intuitive for users of all levels.
